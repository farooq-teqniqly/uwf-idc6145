\documentclass{article}
\usepackage{graphicx} % Required for inserting images
\usepackage{listings}
\usepackage{xcolor}
\usepackage{geometry}
\usepackage{parskip}
\usepackage{amsmath}
\usepackage{url}

\geometry{
 letterpaper,
 left=20mm,
 top=20mm,
 }
 
\graphicspath{{images/}}

\title{\vspace{-2cm}Temporal Congestion Analysis and Predictive Modeling of Ride-Hailing Trips in NYC Using High-Volume FHV Data}
\author{Farooq Mahmud}
\date{April 25, 2025}

\definecolor{backcolor}{rgb}{0.95,0.95,0.92}

\lstdefinestyle{mystyle}{
    backgroundcolor=\color{backcolor},
    captionpos=b                  
}

\lstset{style=mystyle}

\begin{document}

\maketitle

\section{Abstract}
This paper explores traffic congestion patterns in New York City by analyzing High-Volume For-Hire Vehicle (FHV) trip records from services like Uber and Lyft. Using data from all 12 months of 2024, we investigate how trip durations and speeds vary by time of day, day of week, and provider. After extensive preprocessing in PySpark, we identify temporal trends and build predictive models in R using linear and logistic regression. The study demonstrates that the dataset size and analysis complexity necessitate big data tools for processing and modeling at scale.


\section{Introduction}
Urban traffic congestion continues to be a pervasive problem affecting the efficiency of transportation systems, contributing to environmental concerns, and impacting commuter well-being. In densely populated cities like New York, where ride-hailing services such as Uber and Lyft operate at massive scale, understanding when and where congestion occurs is critical for optimizing urban mobility.
This project investigates temporal patterns in congestion using High-Volume For-Hire Vehicle (FHV) trip data collected and published by the NYC Taxi and Limousine Commission (TLC). With over 200 million trip records from 2024 alone, the dataset provides a rich source of behavioral and operational insights. The central research question is: How do trip duration and speed, as proxies for congestion, vary across different times of day, days of the week, and ride-hailing providers?
The objectives of the study are to (1) identify temporal congestion trends through descriptive analysis, and (2) develop regression models to predict trip duration or classify trips as congested based on their features. Given the volume and granularity of the dataset, traditional analysis tools fall short, necessitating the use of big data processing technologies like Apache Spark. This research aims to produce interpretable insights that can inform urban planning, traffic policy, and ride-hailing operations.

\section{Background Review}

A growing body of research has explored the use of ride-hailing data for understanding urban transportation dynamics. Previous studies have focused on topics such as demand prediction, ride-sharing optimization, pricing strategy, and transportation equity. NYC’s Taxi and Limousine Commission has provided open access to large-scale trip data, which has enabled a range of analyses from temporal ridership patterns to spatial clustering of pick-up and drop-off zones \cite{nyctlc2023}.
Several works have leveraged this dataset to evaluate ride-hailing’s impact on traffic flow and public transit usage. For example, a study by NYU Tandon found that New York City’s congestion surcharge reduced the number of ride-hailing trips but did not significantly alleviate congestion \cite{nyu2023}. Additionally, research from the CUNY Graduate Center examined the 2019 congestion surcharge’s effects on FHV usage in Manhattan, showing a decline in rides from charged zones \cite{cuny2023}.
Other studies have investigated surge pricing’s influence on driver behavior \cite{surgepricing2022} and analyzed pay disparities across platforms using data from 2019 to 2022 \cite{ucla2023}. However, many of these analyses are limited in scope—often examining only a few months of data, focusing exclusively on yellow taxi or green cab datasets, or lacking predictive components.
There remains a gap in literature concerning sustained, full-year temporal analysis of congestion based specifically on High-Volume FHV data. Furthermore, predictive modeling using such data is less common, particularly models focused on estimating congestion or trip delays. Our project addresses these shortcomings by utilizing a complete 12-month dataset and incorporating regression modeling. Additionally, our use of big data processing tools distinguishes this study from prior work that relies on single-machine or static data approaches.

\section{Conclusion}
This study highlights the feasibility and value of using big data tools to analyze ride-hailing trip data for congestion insights. It confirms the need for scalable data processing when working with multi-year FHV datasets and shows that simple predictive models can effectively estimate congestion levels using temporal features. Future work could include incorporating spatial features, extending modeling to multi-year data, or testing real-time streaming analysis.

\bibliographystyle{plain}
\bibliography{refs}

\end{document}
